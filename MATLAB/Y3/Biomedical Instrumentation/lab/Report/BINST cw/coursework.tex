\documentclass[12pt,twoside]{article}
\usepackage[table]{xcolor}
\definecolor{light-gray}{gray}{0.9}
\definecolor{dark-gray}{gray}{0.5}

\newcommand{\reporttitle}{Biomedical Instrumentation Lab Report}
\newcommand{\reportauthor}{Luis Chaves Rodriguez}
\newcommand{\reporttype}{}
\newcommand{\cid}{01128684}

% include files that load packages and define macros
\input{includes} % various packages needed for maths etc.
\input{notation} % short-hand notation and macros


%%%%%%%%%%%%%%%%%%%%%%%%%%%%

\begin{document}
% front page
\input{titlepage}


%%%%%%%%%%%%%%%%%%%%%%%%%%%% Main document
\section{Lab 1 - Passive Filters}
\textbf{Question 1.1.} \textit{What function does this circuit perform? Can you identify any possible sub---circuits
within it? Write and elaborate on the basic relations needed for deriving the transfer
function of this block.}\\\\
This circuit performs a band-pass filter action. It is composed of four passive filters, two first order low-pass RC (Resistor and Capacitor) filters with a cut-off frequency of 4.08 Hz on the left-hand side of the circuit and two first-order high-pass RC filters with a cut-off frequency of 80 mHz. The cut-off frequency of such filters can be calculated from:$$ f_{c-L} = f_{c-H} = {1 \over 2\pi RC}$$

To derive the transfer function of any of these filters we can first take the impedance of the components we are dealing with:
$$Z_{capacitor} = {1\over Cs}$$
$$Z_{resistor} = R$$
We can easily see how the circuit for every filter [comprised by one capacitor and one resistor at a time] is a voltage divider.
Where, $${V_{out} \over V_{in} }={ {Z_{capacitor}}\over {Z_{capacitor}+Z_{resistor}}}\qquad \text {for the low-pass filter.}$$ 
$${V_{out} \over V_{in} }={ {Z_{resistor}}\over {Z_{capacitor}+Z_{resistor}}}\qquad \text{for the high-pass filter.}$$ 
Through further expanding and simplifying these equations
the transfer function of a low-pass filter of the kind we use in the circuit is found to be:$$ L(s) = {1 \over RCs+1}$$
And the transfer function of a high-pass filter of the kind we use in the circuit is: $$ H(s) = {RCs \over RCs+1}$$
, where s is the Laplace variable (for the purpose of our studies s = j$\omega$ [angular frequency]).
$$$$The transfer function of the whole circuit results in the multiplication of the transfer function of the four sub-circuits that compose it.
\newpage

\textbf{Measurements 1 and 2}\\\\
{\rowcolors{2}{light-gray}{dark-gray}
\centering
 \begin{tabular}{|c |c |c |c |c|} 
 \hline
Frequency [Hz] & Output Voltage (M1)  &Output Voltage (M2) & dB drop(M1)&dB drop(M2) \\ 
 \hline
 0.5& 0.936 & 0.896 & -0.574 & -0.954 \\ 
 \hline
 1 & 0.84 & 0.792 & -1.514&-2.025 \\
 \hline
 2 & 0.672 & 0.588 & -3.453 &-4.612 \\
 \hline
 3 & 0.568 & 0.432 & -4.913&-7.290 \\
 \hline
 4 & 0.488 & 0.332 & -6.232 &-9.577 \\
 \hline
 5 & 0.44 & 0.276 & -7.131&-11.18 \\
 \hline
 6 &0.408 & 0.224 & -7.787&-12.99 \\
 \hline
 7 & 0.368 & 0.192 & -8.683&-14.33 \\
 \hline
 8 & 0.348&0.138&-9.168&-15.70\\
 \hline
 9 & 0.332&0.138&-9.577&-17.20 \\
 \hline
 10 & 0.31 & 0.122 & -10.17&-18.27 \\
 \hline
\end{tabular}
}\\\\

\textbf{Question 2-b} \textit{If the circuit was extended as shown in the diagram below how manydBs/Oct attenuation would you expect between the frequencies of 12Hz and24Hz?}
\\\\
In the scenario where we are considering frequencies of 12 Hz and 24 Hz, it is safe to assume that the dB drop of the high-pass filter will be 0 (i.e. there will be no signal attenuation by the high-pass filter). Hence our transfer function can simplify the product of the low-pass filter sub-circuits transfer functions.\\
By adding a third LP filter in our circuit we get:
$$TF = {V_{out}\over V_{in}} = {\big({1 \over {1+RCs}}\big)^3}$$
We could calculate the decibel drop in amplitude by first calculating the gain(magnitude) of this transfer function:
$$ \left\Vert{V_{out}\over V_{in}}\right\Vert = \left\Vert{\big({1 \over {1+RCs}}\big)^3}\right\Vert$$
$$\Leftrightarrow \left\Vert{V_{out}\over V_{in}}\right\Vert = \left\Vert{\big({1 \over {1+RCs}}\big)}\right\Vert^3$$
Now, we substitute s for j$\omega$ and take the magnitude at the numerator and denominator:
$$\Leftrightarrow \left\Vert{V_{out}\over V_{in}}\right\Vert = \sqrt{1^2 \over {1^2+{\big({\omega\over\omega_o}\big)}^2}}^3 \text{,by taking $\omega_o = {1\over RC}$}$$
In the next step we convert the gain $\left\Vert{V_{out}\over V_{in}}\right\Vert$ into decibels:
$$ dB = 20\:log_{10}\left(\left\Vert{V_{out}\over V_{in}}\right\Vert\right)$$
$$\Leftrightarrow dB = 20\:log_{10}\left(\sqrt{1 \over {1+{\big({\omega\over\omega_o}\big)}^2}}^3\right)$$
$$\Leftrightarrow dB = 60\:log_{10}(1)-60log_{10}\sqrt{1+\big({\omega\over\omega_o}\big)^2}$$
$$\Leftrightarrow dB = -30\:log_{10}\big(1+\big({\omega\over\omega_o}\big)^2\big)$$
By changing $\omega$ for 2$\pi$f, and finally the frequency f for 12 Hz and 24 Hz and R and C for their respective values we get:
$$\omega_o = {1\over{39\cdot10^-3}} \:rad\cdot s^{-1} = 25.6\:rad\cdot s^{-1} $$
$$ f = 12 Hz \qquad  dB_{1} =  -30\:log_{10}\big(1+\big({2\pi\cdot 12\over25.6}\big)^2\big)$$
$$\Leftrightarrow dB_{1} = -29.6\: dBs$$
$$ f = 24 Hz \qquad dB_{2} =  -30\:log_{10}\big(1+\big({2\pi\cdot 24\over25.6}\big)^2\big)$$
$$\Leftrightarrow dB_{2} = -46.6 \:dBs$$
The increase in frequency from 12 Hz to 24 Hz is the equivalent to one octave hence
we expect an attenuation between the frequencies of 12 Hz and 24 Hz of:
$$ dB\:drop = {{dB_{2}-dB_{1}} \over {1}}= 17\: dBs/Oct $$
\newpage

\section{Basics}

%\subsection{Figures}
%A figure can be included as follows:
%\begin{figure}[b]
%\centering % this centers the figure
%\includegraphics[width = 0.7\hsize]{./figures/imperial} % this includes the figure and specifies that it should span 0.7 times the horizontal size of the page
%\caption{This is a figure.} % caption of the figure
%\label{fig:imperial figure} % a label. When we refer to this label from the text, the figure number is included automatically
%\end{figure}
%Fig.~\ref{fig:imperial figure} shows the Imperial College logo. 

Some guidelines:
\begin{itemize}
\item Always use vector graphics (scale free)
\item In graphs, label the axes
\item Make sure the font size (labels, axes) is sufficiently large
\item When using colors, avoid red and green together (color blindness)
\item Use different line styles (solid, dashed, dotted etc.) and different markers to make it easier to distinguish between lines
\end{itemize}

%\subsection{Notation}
%\begin{table}[tb]
%\caption{Notation}
%\label{tab:notation}
%\centering
%\begin{tabular}{ll}
%Scalars & $x$\\
%Vectors & $\vec x$\\
%Matrices & $\mat X$\\
%Transpose & $\T$\\
%Inverse & $\inv$\\
%Real numbers & $\R$\\
%Expected values & $\E$\\
%\end{tabular}
%\end{table}
%Table~\ref{tab:notation} lists some notation with some useful shortcuts (see latex source code).

\subsubsection{Equations}
Here are a few guidelines regarding equations
\begin{itemize}
\item Please use the \texttt{align} environment for equations (\texttt{eqnarray} is buggy)
\item Please number all equations: It will make things easier when we need to refer to equation numbers. If you always use the \texttt{align} environment, equations are numbered by default.
\item Vectors are by default column vectors, and we write 
\begin{align}
\vec x &= \colvec{1,2}
\end{align}
\item Note that the same macro (\texttt{$\backslash$colvec}) can produce vectors of variable lengths, as
\begin{align}
\vec y &= \colvec{1,2,3,4}
\end{align}
\item Matrices can be created with the same command. The \& switches to the next column:
\begin{align}
\mat A = \begin{bmatrix}
1 & 2 & 3\\
3 & 4 & 5
\end{bmatrix}
\end{align}
\item Determinants. We provide a simple macro (\texttt{$\backslash$matdet}) whose argument is just a matrix array:
\begin{align}
\matdet{
1 & 2 & 3\\
3 & 4 & 5\\
2 & 2 & 2
}
\end{align}
\item If you do longer manipulations, please explain what you are doing: Try to avoid sequences of equations without text breaking up. Here is an example:
We consider
\begin{align}
U_1 = [\colvec{1,1,0,0},\, \colvec{0,1,1,0},\, \colvec{0,0,1,1}]
\subset\R^4, \quad 
U_2 = [\colvec{-1,1,2,0},\, \colvec{0,1,0,0}]
\subset\R^4\,.
\end{align}
To find a basis of $U_1\cap U_2$, we need to find all $\vec x \in V$ that can be represented as linear combinations of the basis vectors of $U_1$ and $U_2$, i.e., 
\begin{align}
\sum_{i=1}^3 \lambda_i \vec b_i = \vec x = \sum_{j=1}^2 \psi_j \vec c_j\,,
\end{align}
where $\vec b_i$ and $\vec c_j$ are the basis vectors of $U_1$ and $U_2$, respectively.
%
The matrix $\mat A = [\vec b_1|\vec b_2|\vec b_3| -\vec c_1|-\vec
c_2]$ is given as
\begin{align}
\mat A = 
\begin{bmatrix}
1 & 0 & 0 & 1 & 0\\
1 & 1 & 0 & -1 & -1\\
0 & 1 & 1 & -2 & 0\\
0 & 0 & 1 & 0 & 0
\end{bmatrix}\,.
\end{align}
By using Gaussian elimination, we determine the corresponding reduced row echelon form 
\begin{align}
\begin{bmatrix}
1 & 0 & 0 & 1& 0\\
0 & 1 & 0 & -2 & 0\\
0 & 0 & 1 & 0 & 0\\
0 & 0 & 0 & 0 & 1
\end{bmatrix}
\,.
\end{align}
We keep in mind that we are interested in finding $\lambda_1,\lambda_2,\lambda_3\in\R$ and/or $\psi_1,\psi_2\in\R$ with 
\begin{align}
\begin{bmatrix}
1 & 0 & 0 & 1& 0\\
0 & 1 & 0 & -2 & 0\\
0 & 0 & 1 & 0 & 0\\
0 & 0 & 0 & 0 & 1
\end{bmatrix}
\colvec{\lambda_1, \lambda_2, \lambda_3, \psi_1, \psi_2}
=\vec 0\,.
\end{align}
From here, we can immediately see that $\psi_2=0$ and $\psi_1\in\R$ is
a free variable since it corresponds to a non-pivot column, and our solution is 
\begin{align}
U_1\cap U_2 = \psi_1\vec c_1 =  [ \colvec{-1,1,2,0} ]
\,, \quad \psi_1\in\R\,.
\end{align}
\end{itemize}


\subsection{Gaussian elimination}
We provide a template for Gaussian elimination. It is not perfect, but it may be useful:

\begin{elimination}[6]{5}{8mm}{1}
    \eliminationstep
    {
1 & - 2 & 1 & -1 & 1 &  0\\     
0 & 0 & -1 & 1 & -3 & 2\\
0 & 0 & 0 & -3 & 6 & -3\\
0 & 0 & -1 & -2 & 3 & a
    }
    {
\\
\\
\\
-R_2
    }
    \\
\eliminationstep
 {
1 & - 2 & 1 & -1 & 1 &  0\\     
0 & 0 & -1 & 1 & -3 & 2\\
0 & 0 & 0 & -3 & 6 & -3\\
0 & 0 & 0 & -3 & 6 & a-2
    }
    {
\\
\\
\\
-R_3
    }\\
\eliminationstep{
1 & - 2 & 1 & -1 & 1 &  0\\     
0 & 0 & -1 & 1 & -3 & 2\\
0 & 0 & 0 & -3 & 6 & -3\\
0 & 0 & 0 & 0 & 0 & a+1
}
{
\\
\cdot (-1)\\
\cdot (-\tfrac{1}{3})\\
\\}
\\
\eliminationstep{
1 & - 2 & 1 & -1 & 1 &  0\\     
0 & 0 & 1 & -1 & 3 & -2\\
0 & 0 & 0 & 1 & -2 & 1\\
0 & 0 & 0 & 0 & 0 & a+1
}{}
\end{elimination}

The arguments of this environment are:
\begin{enumerate}
\item Number of columns (in the augmented matrix)
\item Number of free variables (equals the number of columns after which the vertical line is drawn)
\item Column width
\item Stretch factor, which can stretch the rows further apart.
\end{enumerate}



\newpage
\section{Answer Template}
\begin{enumerate}[1)]
\item Discrete models

\begin{enumerate}[a)]
\addtocounter{enumii}{2} % change to enumi if you use sections rather than enumerate for question numbers
\item 
\item
\item 
\end{enumerate}


\item Differentiation

\begin{enumerate}[a)]
\item 
\item
\addtocounter{enumii}{1} 
\item 
\item
\end{enumerate}


\item Continuous Models

\begin{enumerate}[a)]
\item 
\item
\item 
\item 
\item
\item 
\item 
\end{enumerate}

\item Linear Regression


\begin{enumerate}[a)]
\item 
\item
\item 
\item 
\end{enumerate}

\item Ridge Regression


\begin{enumerate}[a)]
\item 
\item
\item 

\begin{enumerate}[i)]
\item 
\item
\end{enumerate}

\end{enumerate}

\item Bayesian Linear Regression


\begin{enumerate}[a)]
\addtocounter{enumii}{1} 
\item 
\item
\item 
\item (bonus)
\end{enumerate}

\end{enumerate}














\end{document}
%%% Local Variables: 
%%% mode: latex
%%% TeX-master: t
%%% End: 
